% !TEX TS-program = pdflatex
% !TEX encoding = UTF-8 Unicode

% This is a simple template for a LaTeX document using the "article" class.
% See "book", "report", "letter" for other types of document.

\documentclass[11pt]{article} % use larger type; default would be 10pt

\usepackage[utf8]{inputenc} % set input encoding (not needed with XeLaTeX)

%%% Examples of Article customizations
% These packages are optional, depending whether you want the features they provide.
% See the LaTeX Companion or other references for full information.

%%% PAGE DIMENSIONS
\usepackage{geometry} % to change the page dimensions
\geometry{a4paper} % or letterpaper (US) or a5paper or

\usepackage{graphicx} % support the \includegraphics command and options

% \usepackage[parfill]{parskip} % Activate to begin paragraphs with an empty line rather than an indent

%%% PACKAGES
\usepackage{booktabs} % for much better looking tables
\usepackage{array} % for better arrays (eg matrices) in maths
\usepackage{paralist} % very flexible & customisable lists (eg. enumerate/itemize, etc.)
\usepackage{verbatim} % adds environment for commenting out blocks of text & for better verbatim
\usepackage{subfig} % make it possible to include more than one captioned figure/table in a single float
% These packages are all incorporated in the memoir class to one degree or another...
\usepackage{amsmath,epsfig}
\usepackage{amsthm, amssymb}

%%% HEADERS & FOOTERS
\usepackage{fancyhdr} % This should be set AFTER setting up the page geometry
\pagestyle{fancy} % options: empty , plain , fancy
\renewcommand{\headrulewidth}{0pt} % customise the layout...
\lhead{}\chead{}\rhead{}
\lfoot{}\cfoot{\thepage}\rfoot{}

%%% SECTION TITLE APPEARANCE
\usepackage{sectsty}
\allsectionsfont{\sffamily\mdseries\upshape} % (See the fntguide.pdf for font help)
% (This matches ConTeXt defaults)

%%% ToC (table of contents) APPEARANCE
\usepackage[nottoc,notlof,notlot]{tocbibind} % Put the bibliography in the ToC
\usepackage[titles,subfigure]{tocloft} % Alter the style of the Table of Contents
\renewcommand{\cftsecfont}{\rmfamily\mdseries\upshape}
\renewcommand{\cftsecpagefont}{\rmfamily\mdseries\upshape} % No bold!

\usepackage{ textcomp }

%%% END Article customizations

%%% The "real" document content comes below...

\title{LING773 - Negotiations Project Proposal}
\author{Peng Ye, Youngil Kim, Olivia Buzek}
%\date{} % Activate to display a given date or no date (if empty),
         % otherwise the current date is printed

\begin{document}
\maketitle

\section{Creating N-Grams}

I created n-grams through following process.
\begin{itemize}
\item Preprocessing with raw data - creating \textit{fields\_speaker.txt}: \newline
We analyzed \textit{merged\_turns.csv} and \textit{metadata.csv} for creating \textit{fields\_speaker.txt} file. Because some dialogues do not have \textit{joint profit} in \textit{metadata.csv} file, we should rule out those dialogues even though they exist in \textit{merged\_turns.csv} file. In addition, we rearranaged sequence of fields for better looking; the dyid is positioned as the first item.

\item Tokeninzing with whole dialog files: \newline
The raw data contains many special characters and not yet tokenized for creating n-grams. Basically, we used Treebank tokenization and this can be found in \textit{http://www.cis.upenn.edu/$\sim$treebank/tokenization.html}. Following list shows a part of the tokenization rules.
\begin{itemize}
\item most punctuation is split from adjoining words.
\item verb contractions and the Anglo-Saxon genitive of nouns are split into their component morphemes, and each morpheme is tagged separately.
\begin{itemize}
\item Examples
\begin{itemize}
\item children's \textrightarrow children 's
\item I'm \textrightarrow I 'm
\end{itemize}
\end{itemize}
This tokenization allows us to analyze each component separately.\newline
In addition, we need to correct some words to see more precise result. For example, some speakers omit \textit{question mark} or \textit{apostrophe}, and sometimes there are several special characters which is not interpretable in plain text. You can see the detailed conversion in \textit{tokenize} function in \textit{create\_ngram.py}.
\end{itemize}
\item Count n-grams and Create n-gram ID:\newline
After tokenizing, we count number of n-grams to see the characteristics of each dialogue. For counting ngrams, we used \textit{count.pl} in Ted Pedersen’s Ngram Statics Package with general stop words because stop words can hide real characteristics of n-grams. We seperately count n-grams for Wine, Grocery, and all dialogues for seeing the characteristics of each group of speakers. Also, we count n-grams per each dyad too. \newline
In addition, we created n-gram id files to express ngrams. For example, \textit{grand opening date} which is one of trigram is expressed as \textit{3-0001}. By using this ngram id, we can save memory and storage for analyzing whole data.
\item Make n-gram feature file:\newline
By using counted n-grams, we created n-gram feature file which can be converted into an ARFF format, input format of Weka. Each row in the file contains n-gram information of each dyad. The first column shows the \textit{dyID}, and the following colums show ngram id and the number of ngrams in the dialogues. Following example shows a part of the feature file. 
\begin{itemize}
\item Example
\begin{verbatim}
02,1-0002:30,1-0001:23,1-0040:18,1-0009:16,1-0012:15 ... ...
03,1-0006:30,1-0001:29,1-0048:23,1-0026:21,1-0025:20 ... ...
\end{verbatim}
\end{itemize} 
We made three n-gram feature files for each category, Wine, Grocery, and all dialogues.
\end{itemize}
\end{document}
